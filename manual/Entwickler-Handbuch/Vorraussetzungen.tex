\section{Voraussetzungen}\label{sec:voraussetzungen}

\subsection{Node.js}\label{subsec:node.js2}

Um den Editor zu installieren, muss \emph{Node.js} auf dem System installiert sein.
Für die Installation von \emph{Node.js} gibt es unterschiedliche Installationsanweisungen, abhängig von dem Betriebssystem, auf dem \emph{Node.js} installiert werden soll.
Zu finden sind diese unter \url{https://nodejs.org/en/download/}.

\subsection{NPM / Yarn}\label{subsec:npm-/-yarn}

\includegraphics[scale=0.1]{assets/npm}
\includegraphics[scale=0.1]{assets/yarn}

Um den Editor zu installieren, muss \emph{NPM} oder \emph{Yarn} installiert sein.
\emph{NPM} wird mit \emph{Node.js} installiert.//

\emph{Yarn} kann unter \url{https://yarnpkg.com/en/docs/install} installiert werden, oder mit \emph{NPM} installiert werden, indem der Befehl \cmd{npm install -g yarn} ausgeführt wird.

NPM und Yarn werden für die Installation der Abhängigkeiten des Editors benötigt.
Sie sind beide Paketmanager für \emph{Node.js}.

\subsection{Vorbereitung}\label{subsec:vorbereitung}

\begin{enumerate}
	\item Lade den Quellcode des Editors herunter.
	\item Entpacke den Quellcode des Editors.
	\item Öffne ein Terminal.
	\item Navigiere in das Verzeichnis, in dem der Quellcode des Editors entpackt wurde.
	\item Führe den Befehl \cmd{npm install} oder \cmd{yarn install} aus.
	\item Warte, bis die Abhängigkeiten installiert wurden.
	Das kann einige Minuten dauern.
	Es kann sein, dass bei der Installation der Abhängigkeiten Warnungen oder Fehler auftreten.
	Diese können ignoriert werden.
	\item Führe den Befehl \cmd{npm i -g ts ts-node} oder \cmd{yarn global add ts ts-node} aus.
	Dieser Befehl installiert die \emph{TypeScript}-Compiler, welche für die Entwicklung des Editors benötigt werden.
	\item Führe den Befehl \cmd{npm run start} oder \cmd{yarn start} aus.
	\item Nun, sollte der, Editor gestartet werden und als ein Fenster angezeigt werden.
\end{enumerate}

Dies ist auch alles auch noch einmal in der \emph{README}-Datei des Editors beschrieben.
