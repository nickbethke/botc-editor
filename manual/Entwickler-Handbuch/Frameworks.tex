\section{Frameworks}\label{sec:frameworks}

Der Editor ist in der Programmiersprache \emph{JavaScript}/\emph{TypeScript} und den \emph{Node.js}-Frameworks \emph{Electron} und \emph{React} geschrieben.\\
Das Styling des Editors wird mit \emph{Tailwind CSS} realisiert.\\\\
In der \href{https://gitlab.uni-ulm.de/softwaregrundprojekt/2022-2023/messe/editor/messe-editor-team-11/-/blob/main/package.json}{\texttt{package.json}}-Datei sind alle Abhängigkeiten des Editors aufgelistet unter \emph{dependencies} und \emph{devDependencies}.

\subsection{JavaScript/TypeScript}\label{subsec:javascript/typescript}
% two images side by side
\includegraphics[scale=0.05]{assets/javascript}
\includegraphics[scale=0.05]{assets/typescript}

% detaillierte Beschreibung von JavaScript

\emph{JavaScript}\footnote{\url{https://www.w3schools.com/js/DEFAULT.asp}} ist eine weit verbreitete Programmiersprache, die hauptsächlich für die Entwicklung von interaktiven Webseiten und Webanwendungen verwendet wird. Sie ermöglicht es Entwicklern, Funktionen, Abläufe und Interaktionen in den Browsern der Benutzer einzubetten. \\
\emph{JavaScript} ist eine interpretierte Sprache, was bedeutet, dass der Code zur Laufzeit ausgeführt wird. \emph{JavaScript} ist eine objektorientierte Sprache, was bedeutet, dass \emph{JavaScript} Objekte verwendet, um Daten zu speichern und zu verarbeiten. \\
JavaScript wird seit 1995 von \emph{Netscape} entwickelt und ist eine Open-Source-Programmiersprache.

\emph{TypeScript}\footnote{\url{https://www.typescriptlang.org/}} ist eine Programmiersprache, die auf \emph{JavaScript} basiert.
\emph{TypeScript} ist eine objektorientierte Sprache, was bedeutet, dass \emph{TypeScript} Objekte verwendet, um Daten zu speichern und zu verarbeiten.
\emph{TypeScript} ist eine kompilierte Sprache, was bedeutet, dass der Code vor der Ausführung in \emph{JavaScript} kompiliert wird.
\emph{TypeScript} ist eine typisierte Sprache, was bedeutet, dass \emph{TypeScript} Variablen und Funktionen einen Typ haben.
TypeScript wurde 2012 von \emph{Microsoft} entwickelt und ist eine Open-Source-Programmiersprache.
\newpage

\subsection{Node.js}\label{subsec:node.js}
\includegraphics[scale=0.1]{assets/nodejs}

\emph{Node.js}\footnote{\url{https://nodejs.org/en/}} ist eine Open-Source-Plattform, die es Entwicklern ermöglicht, serverseitige Anwendungen mit \emph{JavaScript} zu erstellen.
\emph{Node.js} basiert auf der \emph{V8 JavaScript Engine}\footnote{\url{https://v8.dev/}} von \emph{Google}.
\emph{Node.js} ist eine kompilierte Plattform, was bedeutet, dass der Code vor der Ausführung in \emph{JavaScript} kompiliert wird.
Es wurde 2009 von \emph{Ryan Dahl} entwickelt und ist eine Open-Source-Plattform.

\subsection{Electron}\label{subsec:electron}

\includegraphics[scale=0.066]{assets/electron}

\emph{Electron}\footnote{\url{https://electronjs.org/}} ist ein Open-Source-Framework, das es Entwicklern ermöglicht, Desktop-Anwendungen mit \emph{JavaScript}, \emph{HTML} und \emph{CSS} zu erstellen.
\emph{Electron} basiert auf \emph{Node.js} und \emph{Chromium}\footnote{\url{https://www.chromium.org/}}.
\emph{Electron} ist eine kompilierte Plattform, was bedeutet, dass der Code vor der Ausführung in \emph{JavaScript} kompiliert wird.
Es wurde 2013 von \emph{GitHub} entwickelt und ist ein Open-Source-Framework.

\newpage

\subsection{React}\label{subsec:react}

\includegraphics[scale=0.1]{assets/react-logo}

\emph{React}\footnote{\url{https://reactjs.org/}} ist ein Open-Source-Framework, das es Entwicklern ermöglicht, Benutzeroberflächen mit \emph{JavaScript} und \emph{HTML} zu erstellen.
Es vereinfacht die Entwicklung von Benutzeroberflächen, indem es Komponenten verwendet, um die Benutzeroberfläche in unabhängige, wiederverwendbare Teile zu unterteilen.
Es wird seit 2013 von \emph{Facebook} entwickelt und ist ein Open-Source-Framework.

\subsection{Tailwind CSS}\label{subsec:tailwind-css}

\includegraphics[scale=0.5]{assets/tailwindcss-logotype}

\emph{Tailwind CSS}\footnote{\url{https://tailwindcss.com/}} ist ein Open-Source-Framework, das es Entwicklern ermöglicht, Benutzeroberflächen mit \emph{CSS} zu erstellen.
Es vereinfacht die Entwicklung von Benutzeroberflächen, indem es Klassen verwendet, um die Benutzeroberfläche in unabhängige, wiederverwendbare Teile zu unterteilen.
Es wurde 2017 von \emph{Adam Wathan} entwickelt und ist ein Open-Source-Framework.
